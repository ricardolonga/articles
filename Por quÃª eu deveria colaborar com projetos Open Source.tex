\documentclass[a4paper,11pt]{article}
\usepackage[T1]{fontenc}
\usepackage[utf8]{inputenc}
\usepackage{lmodern}

\title{Por quê eu deveria colaborar para projetos Open Source?}
\author{Daniel Cunha (soro) \\ danielsoro@gmail.com}

\begin{document}

\maketitle
\newpage
\tableofcontents
\newpage

\section{O que é Open Source?}
Bom, muitas pessoas confudem Open Source com Software Livre, quando falamos de Open Source, estamos dizendo que o código é aberto, mas nem sempre ele é livre. Ou seja, se eu tenho um código Open Source, ele não necessáriamente é Software Livre. Já no caso inverso, se tenho um software livre, ele também será Open Source.\newline\newline
''Qualquer licença de software livre é também uma licença de código aberto (Open Source), mas o contrário nem sempre é verdade''.

\section{Vantagens de contribuir pra um projeto Open Source?}
  Muitas pessoas se perguntam por quê perderiam o seu precioso tempo para contribuir em projetos open source, vamos ver o que isso pode nos trazer de bom.\newline\newline
Dependendo do seu ponto de vista, open source não é trabalhar de graça. Imagine você tendo um retorno de conhecimento fora do comum. Isso não paga o seu trabalho? Bom, eu penso que o meu sim. Perder horas de estudo vai me enriquecer intelectualmente, vai me fazer melhorar no trabalho (meu ganha pão) e pode me dar outras oportunidades fora do comum.\newline\newline
Como assim Daniel?
Nesse ambiente você desenvolvedor ajuda outros desenvolvedores (levando para o nosso lado profissional). Participando desses projetos você pode aprender muito. Imagine você contribuindo para o Yougi. Você terá um conhecimento maior sobre como implementar utilizando Java EE. Configurar certas funcionalidades no WildFly e etc.\newline\newline
Ler o código existente já vai te dá uma bagagem incrível, começar a entender o código, como ele funciona, como são utilizados os frameworks e etc. Tudo isso já vai te dá uma bagagem muito boa, ou seja, você vai estar aprendendo coisas novas ou melhorando as que já conhece.O importante é colaborar.

\section{Como eu faço para colaborar?}
\subsection{Reportando bugs}
Sim, mais do que importante, reportar um problema é uma forma de contribuir. Imagine se você acha um erro, que é reproduzido de uma forma específica que ninguém nunca fez na vida. Esse projeto manterá esse erro durante um bom tempo e as vezes podem ser bem críticos. Reportar erros é sim uma forma de contribuir, você está ajudando na evolução do projeto, seja codificando ou testando.\newline\newline
Geralmente todo mundo começa reportando bugs, após se inscrever na lista, se apresentar. Geralmente o primeiro passo é reportar os erros que você achou. Mas, esse passo requer alguma atenção também:
\begin{itemize}
  \item Confira as issues existentes antes, evita criar issues duplicadas.
  \item Seja ativo na lista de email, seus feedbacks lá também são importante.
  \item Sempre tente explanar ao máximo como você conseguiu reproduzir o erro.
  \item Nunca reporta falha de segurança de forma pública, nesse caso procure o owner do projeto ou verifique se o projeto não tem uma lista dedicada para esse tipo de problema.
\end{itemize}
\subsection{Corrigindo issue}
Depois que você já estiver mais tranquilo em relação ao projeto, que já entende o fluxo, está na hora de começar a corrigir as issues. Então, eleja aquela que você acha que consegue fazer e comece a trabalhar nela.
\subsection{Participe das discussões}
Como citei em um dos bullets acima na section de "Reportando Bugs", estar presente nas duscussões é sempre bom. As vezes as pessoas tem dúvidas na lista e talvez você saiba como ajudar. As vezes as pessoas precisam que outras testem o que ele fez, talvez nesse momento você também posso ajudar e o seu feedback é muito importante para o projeto.
\subsection{Revisar outras colaborações}
Sim, você mesmo pode revistar outras colaborações, talvez você consigua refatorar um código ou até indicar uma implementação melhor. Agora você entende o que eu quis dizer quando falei: Desenvolvedores ajudando Desenvolvedores. A questão do suporte é MUITO válida para esses projetos.\newline\newline
Nunca esqueça que nesse mundo não existe contribuição \textbf{pequena demais}, existe inúmeras formas de participar e \textbf{NUNCA} tenha medo de pedir ajuda, projetos assim, visam ter pessoas interessadas em ajudar, se seu interesse é ajudar no projeto e você ainda precisa de ajuda, não tenha medo de pedi-la, todos acreditam que você pode ser um forte candidato a trazer coisas boas para o projeto.
\end{document}
